\documentclass[12pt]{article}
% We can write notes using the percent symbol!
% The first line above is to announce we are beginning a document, an article in this case, and we want the default font size to be 12pt
\usepackage[utf8]{inputenc}
% This is a package to accept utf8 input.  I normally do not use it in my documents, but it was here by default in Overleaf.
\usepackage{amsmath}
\usepackage{amssymb}
\usepackage{amsthm}
% These three packages are from the American Mathematical Society and includes all of the important symbols and operations 
\usepackage{fullpage}
% By default, an article has some vary large margins to fit the smaller page format.  This allows us to use more standard margins.
\usepackage{graphicx}

\setlength{\parskip}{1em}
% This gives us a full line break when we write a new paragraph


\begin{document}
% Once we have all of our packages and setting announced, we need to begin our document.  You will notice that at the end of the writing there is an end document statements.  Many options use this begin and end syntax.

\graphicspath{{figures/}}

\begin{center}
    \Large     Tarea 3 \\
    Nahuel Almeira
    \normalsize
\end{center}

\section{Ecuaci\'on}

En esta pr\'actica, resolveremos la ecuaci\'on de onda en una dimensi\'on, con velocidad dependiente de la posici\'on. La ecuaci\'on correspondiente es 

\begin{equation}
\phi_{tt} = (v(x)^2\phi_x)_x, \quad v(x) = 1 + \lambda \sin(\omega x),\quad (x,t) \in [0,1]\times [0.\infty)
\end{equation}

Para resolver num\'ericamente, primero definimos las variables

\begin{align}
V &= \phi_x\\
U &= \phi_t.
\end{align}

De este modo, la ecuaci\'on se puede transformar en un sistema de dos ecuaciones diferenciales acopladas de primer orden:

\begin{equation}
\begin{pmatrix}
V_t \\
U_t
\end{pmatrix} = 
\begin{pmatrix}
0 & 1 \\
1 & 0
\end{pmatrix}
\begin{pmatrix}
(v(x) V)_x \\
U_x
\end{pmatrix},
\end{equation}

que se puede expresar tambi\'en como

\begin{align}
V_t &= U_x, \\
U_t &= (v^2(x) V)_x .
\end{align}

La energ\'ia del sistema est\'a dada por la expresi\'on

\begin{equation}
E = \dfrac{1}{2}\int (v^2(x) \phi_x^2 + \phi_t^2) dx = \dfrac{1}{2} \int_0^1 (v^2(x) V^2 + U^2) dx.
\end{equation}

Veamos que la misma se conserva:

\begin{align}
\dot{E} &=  \int_0^1 (v^2\phi_x \phi_{xt} + \phi_t \phi_{tt}) dx \nonumber\\
&=  \int_0^1 (v^2 \phi_x \phi_t)_x dx \nonumber \\
&=  v^2 \phi_x \phi_t \bigg|_0^1 = 0
\end{align}

\begin{align}
\dot{E} &=  \int_0^1 (v^2 V U_x + U U_t) dx \nonumber\\
\dot{E} &=  \int_0^1 (v^2 V U_x + U (v^2 V)_x) dx \nonumber\\
&=  \int_0^1 (v^2 U V)_x dx \nonumber \\
&=  v^2 U V \bigg|_0^1 = 0
\end{align}

\section{Implementaci\'on num\'erica}

Para resolver la ecuaci\'on, utilizaremos dos implementaciones diferentes. En ambos casos utilizaremos el m\'etodo de l\'ineas, con un RK4 para la derivada temporal. Lo que cambiar\'a en cada caso, ser\'a la implementaci\'on de la derivada espacial. 

\subsection{Caso \textbf{STRICT}}

La aproximaci\'on es la siguiente:

\begin{align}
V_t &= D_x(U) \\
U_t &= D_x (v^2 V),
\end{align}

donde $D_x$ representa alg\'un operador de derivadas espaciales. Al igual que en la Tarea 2, utilizaremos distintos \'ordenes de aproximaci\'on.



\subsection{Caso \textbf{NON-STRICT}}



La aproximaci\'on es la siguiente:

\begin{align}
V_t &= D_x(U) \\
U_t &= 2 v v_x V + v^2 D_x(V),
\end{align}

donde $v_x = \lambda \omega \cos(\omega t)$. Es decir que en este caso aplicamos la regla del producto para distribuir el operador de derivadas espaciales. Como veremos en la siguiente secci\'on, el hacer esto introduce un error en la soluci\'on num\'erica.

\section{Energ\'ia num\'erica}

Al igual que en la Tarea 2, haremos un desarrollo considerando, por simplicidad, condiciones de contorno peri\'odicas y el operador de derivadas espaciales $D_X = D_0$. Este mismo an\'alisis puede realizarse para otras condiciones de contorno y para operadores de mayor orden.

Utilizando la regla del trapecio para integrar, la energ\'ia num\'erica del sistema est\'a dada por 

\begin{equation}
E(t) = \dfrac{h}{2}\sum_{j=1}^{N-1} (v_j^2 V_j^2 + U_j^2) + \dfrac{h}{4} (v_0^2 V_0^2 + U_0^2) + \dfrac{h}{4} (v_N^2 V_N^2 + U_N^2). 
\end{equation}

Por periodicidad, se tiene

\begin{equation}
E(t) = \dfrac{h}{2}\sum_{j=0}^{N-1} (v_j^2 V_j^2 + U_j^2) 
\end{equation}

Derivando respecto al tiempo,

\begin{align}
\dot{E}(t) &= h \sum_{j=0}^{N-1} \bigg[v_j^2 V_j (V_j)_t + U_j (U_j)_t\bigg]. \nonumber \\
\dot{E}(t) &= h \sum_{j=0}^{N-1} \bigg[v_j^2 V_j D_0(U_j) + U_j (U_j)_t\bigg]. 
\end{align}

Veamos si la energ\'ia se conserva en cada una de las discretizaciones. En el caso \textbf{STRICT}, tenemos


\begin{align}
\dot{E}_{\mathrm{S}}(t) &= h \sum_{j=0}^{N-1} \bigg[v_j^2 V_j D_0(U_j) + U_j (v_j^2 (V_j)_x)_x\bigg] \nonumber \\
 &= h \sum_{j=0}^{N-1} \bigg[v_j^2 V_j D_0(U_j) + U_j D_0(v_j^2V_j)\bigg] \nonumber \\
&= (D_0 U, v^2V)_{0, N-1} + (U, D_0(v^2 V))_{0, N-1} \nonumber\\
&= \dfrac{h}{2} \bigg(U_j v_{j+1}^2V_{j+1} + U_{j+1}v_j^2 V_j \bigg) \bigg|_{N-1}^{N - 1} \nonumber \\
&= 0.
\end{align}

Luego, esta discretizaci\'on conserva la energ\'ia.

Por otro lado, para la discretizaci\'on \textbf{NON-STRICT}, tenemos 

\begin{align}
\dot{E}_{\mathrm{NS}}(t) &= h \sum_{j=0}^{N-1} \bigg[v_j^2 V_j D_0(U_j) + U_j (v_j^2 (V_j)_x)_x\bigg] \nonumber \\
 &= h \sum_{j=0}^{N-1} \bigg[v_j^2 V_j D_0(U_j) + U_j 2 v_j (v_x)_j V_j + U_j v_j^2 D_0(V_j)\bigg] \nonumber \\
&= (D_0 U, v^2V)_{0,N-1} + (v^2D_0( V), U)_{0,N-1} +  h \sum_{j=0}^{N-1} U_j 2 v_j (v_x)_j V_j \nonumber\\
&= (D_0 U, v^2V)_{0,N-1} + (D_0(v^2 V), U)_{0,N-1} - h\sum_{j=0}^{N-1} v_j^2 D_-(V_j) U_{j+1} + h \sum_{j=0}^{N-1} U_j 2 v_j (v_x)_j V_j \nonumber\\
&= - h\sum_{j=0}^{N-1} v_j^2 D_-(V_j) U_{j+1} + h \sum_{j=0}^{N-1} U_j 2 v_j (v_x)_j V_j \nonumber\\
&\neq 0.
\end{align}

%%%%%%%%%%%%%%%%%%%%%%%%%%%%%%%%%%%%%%%%%%%%%%%%%

\begin{thebibliography}{1}

\bibitem{Kreiss-Ortiz} {\em Introduction to Numerical Methods for Time Dependent Differential Equations}, H. Kreiss and O. Ortiz, (2014).

\bibitem{Gustafsson} {\em High Order Difference Methods for Time Dependent PDE}, B. Gustafsson, (2008).

\end{thebibliography}


\end{document}
