\documentclass[12pt]{article}
% We can write notes using the percent symbol!
% The first line above is to announce we are beginning a document, an article in this case, and we want the default font size to be 12pt
\usepackage[utf8]{inputenc}
% This is a package to accept utf8 input.  I normally do not use it in my documents, but it was here by default in Overleaf.
\usepackage{amsmath}
\usepackage{amssymb}
\usepackage{amsthm}
% These three packages are from the American Mathematical Society and includes all of the important symbols and operations 
\usepackage{fullpage}
% By default, an article has some vary large margins to fit the smaller page format.  This allows us to use more standard margins.

\setlength{\parskip}{1em}
% This gives us a full line break when we write a new paragraph


\begin{document}
% Once we have all of our packages and setting announced, we need to begin our document.  You will notice that at the end of the writing there is an end document statements.  Many options use this begin and end syntax.

\begin{center}
    Tarea 2 \\
    Nahuel Almeira
\end{center}

\begin{center}
    \Large - \normalsize
\end{center}

\section{Ecuaci\'on de onda}

En esta pr\'actica resolvemos la ecuaci\'on de onda en dimensi\'on $D = 1$

\begin{equation}
\dfrac{\partial^2 \phi}{\partial t^2} = v^2 \dfrac{\partial^2 \phi}{\partial x^2}.
\end{equation}

La misma pued reducirse a un sistema de dos ecuaciones desacopladas con derivadas de primer orden. Definimos para ello $W \equiv (w_1, w_2)^T$, donde

\begin{align}
w_1 &= \phi_x \\
w_2 &= \phi_t.
\end{align}

As\'i, la ecuaci\'on se puede expresar como

\begin{equation}
\begin{pmatrix}
\partial_t w_1 \\
\partial_t w_2
\end{pmatrix} = 
\begin{pmatrix}
0 & 1 \\
v^2 & 0
\end{pmatrix} \cdot
\begin{pmatrix}
\partial_x w_1 \\
\partial_x w_2
\end{pmatrix}.
\end{equation}

Es decir,

\begin{equation} \label{eq:wave_1d_system}
\partial_t W = A \cdot \partial_x W,\quad 
A = \begin{pmatrix}
0 & 1 \\
v^2 & 0
\end{pmatrix}.
\end{equation}

Para resolver este sistema, debemos diagonalizar la matriz $A$. Para ello, definimos

\begin{equation}
S = \dfrac{1}{\sqrt{2}}
\begin{pmatrix}
v^{-1} & -v^{-1} \\
1 & 1
\end{pmatrix} \quad \text{y} \quad
S^{-1} = \dfrac{1}{\sqrt{2}} 
\begin{pmatrix}
v & 1 \\
-v & 1
\end{pmatrix}.
\end{equation}

Notemos que la matriz $A$ se puede expresar como una matriz diagonal mediante la transformaci\'on 

\begin{equation}
S^{-1} A S = \Lambda = \begin{pmatrix}
v & 0 \\
0 & -v
\end{pmatrix}
\end{equation}

Multiplicando \ref{eq:wave_1d_system} por $S^{-1}$,

\begin{align*}
S^{-1} \partial_t W &= S^{-1} A \partial_x W \\
\partial_t \left( S^{-1} W \right) &= S^{-1} A S S^{-1} \partial_x W \\
\partial_t \left( S^{-1} W \right) &= \Lambda \partial_x \left( S^{-1} W \right), \\
\end{align*}

donde 

\begin{equation}
\left( S^{-1} W \right) = \dfrac{1}{\sqrt{2}} 
\begin{pmatrix}
v w_1 + w_2 \\
-v w_1 + w_2
\end{pmatrix} =
\dfrac{1}{\sqrt{2}} 
\begin{pmatrix}
v \phi_x + \phi_t \\
-v \phi_x + \phi_t
\end{pmatrix}. 
\end{equation}

Definiendo $U$ y $V$ tal que $(U, V)^T = (S^{-1}W)$, obtenemos el sistema diagonal

\begin{equation}
\begin{pmatrix}
V_t \\
U_t
\end{pmatrix} = 
\begin{pmatrix}
v & 0 \\
0 & -v
\end{pmatrix} \cdot
\begin{pmatrix}
V_x \\
U_x
\end{pmatrix}
\end{equation}

Resolveremos el sistema con la condici\'on inicial $V(t=0) = 0$, lo cual implica que $V(t, x) = 0$. Es decir, resolveremos la ecuaci\'on de advecci\'on

\begin{equation}\label{eq:adv}
U_t = - v U_x.
\end{equation}

Adem\'as, simplificaremos el problema definiendo $v = 1$. Utilizaremos condiciones de contorno peri\'odicas en el dominio $x\in [0,1]$ y dos datos iniciales $U(t=0)$. Los datos iniciales son los siguientes:

\begin{itemize}
\item \textbf{Simple Bump:} 

\begin{equation}
U(x, t=0) = (0.25)^{8}(x - 0.25)^4 (x - 0.75)^4.
\end{equation}

\item \textbf{Square Bump:} 

\begin{equation}
U(x, t=0) = 
\begin{cases}
0 & x < 0.25,\\
(0.05)^8(x-0.25)^4 (x-0.35)^4 & 0.25 \leq x \leq 0.3, \\
1 & 0.3 < x < 0.6, \\
(0.05)^8(x-0.65)^4 (x-0.75)^4 & 0.6 \leq x \leq 0.75, \\
0 & 0.75 < x.
\end{cases}
\end{equation}

\end{itemize}

\section{Soluci\'on exacta}

A continuaci\'on utilizamos la teor\'ia de Fourier para hallar la soluci\'on exacta del problema

\begin{align*}
U_t &= v U_x,\quad x\in (0, 1), \quad t > 0 \\
U(x, t=0) &= f(x)\\
U(0, t) &= U(1, t).
\end{align*}

El dato inicial $f(x)$ es tambi\'en 1-peri\'odico, es decir, $f(x+1) = f(x)$. Luego, podemos utilizar su expansi\'on en serie de Fourier

\begin{equation}
U(x,t=0) = f(x) = \sum_{-\infty}^{\infty} e^{2\pi i \omega x} \hat{f}(\omega).
\end{equation}

Aplicando separaci\'on de variables, proponemos el ansatz

\begin{equation}
U(x,t) =  \sum_{-\infty}^{\infty} e^{2\pi i \omega x} \hat{U}(\omega, t).
\end{equation}

Derivando respecto a $x$,

\begin{equation}
U_x(x,t) = \sum_{-\infty}^{\infty} 2\pi i \omega x e^{2\pi i \omega x} \hat{U}(\omega, t).
\end{equation}

Por otro lado, derivando respecto a $t$,

\begin{equation}
U_t(x,t) = \sum_{-\infty}^{\infty} e^{2\pi i \omega x} \hat{U}_t(\omega, t).
\end{equation}

Reemplazando en \ref{eq:adv},

\begin{equation}
\sum_{-\infty}^{\infty} e^{2\pi i \omega x} \hat{U}_t(\omega, t) = \sum_{-\infty}^{\infty} 2\pi i \omega x e^{2\pi i \omega x} \hat{U}(\omega, t).
\end{equation}

Teniendo en cuenta la ortogonalidad de las funciones exponenciales, la ecuaci\'on anterior implica que 

\begin{equation}
\hat{U}_t(\omega, t) = 2 \pi i \omega x \hat{U}(\omega, t), \;\forall \omega.
\end{equation}

Teniendo en cuenta el dato inicial, la ecuaci\'on anterior tiene como soluci\'on

\begin{equation}
\hat{U}(\omega, t) = e^{-2\pi i \omega v t} \hat{f}(\omega).
\end{equation}

Por lo tanto, la soluci\'on al problema es 

\begin{equation}
U(x,t) = \sum_{-\infty}^{\infty} e^{2\pi i \omega (x-vt)} \hat{f}(\omega).
\end{equation}

Es decir, la soluci\'on es una onda viajera que conserva la forma inicial, pero que se desplaza con velocidad constante $v$.

\section{Conservaci\'on de la energ\'ia}

\subsection{Caso anal\'itico}

La energ\'ia se define como 

\begin{equation}
E(t) = \int_0^1 U^2(x,t)\; dx.
\end{equation}

Derivando respecto al tiempo,

\begin{equation} \label{eq:E1}
\dot{E}(t) = 2 \int_0^1 U U_t\; dx = -2 \int_0^1 U U_x \; dx,
\end{equation}

donde utilizamos la ecuaci\'on \ref{eq:adv}. Integrando por partes, obtenemos

\begin{equation} \label{eq:E2}
\dot{E}(t) = -2 U(x,t)\big|_{x=0}^{x=1} + 2 \int_0^1 U_x U \;dx.
\end{equation}

Por las condiciones de contorno, $U(1, t) = U(0,t)$. Luego, de \ref{eq:E1} y \ref{eq:E2} tenemos que $\dot{E}(t) = 0$, por lo que la energ\'ia es constante y, por lo tanto, una cantidad conservada.

\subsection{Caso num\'erico}

Definimos el producto interno eucl\'ideo (y su respectiva norma) para funciones de grilla como

\begin{align}
(u,v)_{\ell, m} &\equiv h \sum_{j=l}^m u_j w_j \\
||u||_{\ell, m}^2 &\equiv (u,u)_{\ell, m}.
\end{align}

Utilizando el m\'etodo del trapecio para integrar, la energ\'ia del sistema puede escribirse como

\begin{equation}
E(t) = h\sum_{j=1}^{N-1} U_j^2 + \dfrac{h}{2} (U_0 + U_N).
\end{equation}

Utilizando la periodicidad, tenemos que $U_0 = U_N$, por lo que podemos simplificar la expresi\'on anterior como

\begin{equation}
E(t) \simeq h\sum_{j=0}^{N-1} U_j^2 = (U,U)_{0,N-1}.
\end{equation}

Consideremos, por simplicidad, la derivada espacial discreta de orden 2 

\begin{equation}
D_0 = \dfrac{D_+ + D_-}{2},
\end{equation}

y veamos que la energ\'ia se conserva (lo mismo puede hacerse para derivadas centradas de \'ordenes superiores). Siguiendo la referencia \cite{Kreiss-Ortiz}, tenemos que 

\begin{equation} \label{eq:product_rule}
(u, D_0 v)_{l,m} + (D_0 u, v)_{l,m} = \dfrac{h}{2} \bigg[ u_j v_{j+1} + u_{j+1} v_j \bigg]\bigg|_{l-1}^m.
\end{equation}

Derivando la energ\'ia,

\begin{equation}
\dot{E}(t) \simeq 2 h \sum_{j=0}^{N-1} U_j \partial_t U_j = -2 h \sum_{j=0}^{N-1} U_j D_0 U_j = -2 (D_0 U, U)_{0,N-1}.
\end{equation}

De acuerdo con \ref{eq:product_rule}, tenemos que 

\begin{align}
\dot{E}(t) &\simeq -\dfrac{h}{2}  \bigg[ U_j U_{j+1} + U_{j+1} U_j \bigg]\bigg|_{-1}^{N-1} \\
&\simeq -h   \bigg[ U_j U_{j+1}\bigg]\bigg|_{N-1}^{N-1} = 0.
\end{align}

\section{Resultados num\'ericos}



\begin{thebibliography}{1}

\bibitem{Kreiss-Ortiz} {\em Introduction to Numerical Methods for Time Dependent Differential Equations}, H. Kreiss and O. Ortiz, (2014).

\end{thebibliography}

\end{document}
